\documentclass{ctexart}
\usepackage[margin=1in]{geometry}
\usepackage{listings}
\usepackage{xcolor}
\usepackage[colorlinks]{hyperref}

\usepackage{verbatim}
\usepackage{color}
\usepackage{url}
\usepackage{enumitem}
\usepackage{graphicx}

\usepackage{amssymb}
\usepackage{amsmath}

\usepackage[T1]{fontenc}
\renewcommand*\familydefault{\sfdefault}
%% Only if the base font of the document is to be sans serif 

\ctexset{
  section/format = \Large\bfseries\raggedright
}

\lstset{
  backgroundcolor = \color{lightgray!30},
  keywordstyle    = \color{blue},
  stringstyle     = \color{purple!40},
  basicstyle      = {\small\ttfamily},
  breaklines      = true,
  tabsize         = 4,
  gobble          = 2,
  numbers         = left,
  numberstyle     = \tiny,
  frame           = single,
  xleftmargin     = \ccwd,
  numbersep       = \ccwd,
%  emphstyle       = {\color{blue}\small\ttfamily},
%  emph            = {mkdir,rmdir,sudo,mount,umount,rm},
}

\title{基于 TeXLive,使用 Sublime Text 3编写{\LaTeX}}
\author{Su H.\thanks{\url{suhang980715@gmail.com}} \ \thanks{腾讯QQ群“{\LaTeX}技术交流5000人群”(群号:91940767)}}

\begin{document}
  
\maketitle

\begin{abstract}
  该文档简单记录一下自己电脑(Windows10系统)上基于 TeXLive,使用 Sublime Text 3编写{\LaTeX}的过程,该过程是基于以下网址对于某些步骤的改进:
  \begin{enumerate}
    \item[(a)]
    \url{https://www.cnblogs.com/fantacity/p/5100434.html}
    % 
    \item[(b)]
    \url{https://blog.csdn.net/qazxswed807/article/details/51234834}
  \end{enumerate}

  \textbf{在安装前,请保证自己的网络可以登陆\href{https://github.com/}{GitHub}网站}

  整个过程大约需要1-2小时。
\end{abstract}
% 
\section{准备工作}
% 
\begin{enumerate}[leftmargin=*]
\item[(a)]
\textbf{下载 TeXLive}

TeXLive是跨平台的{\LaTeX}工具套件,可以在 Linux、 Windows和 Mac OS下使用,包含了{\TeX}以及相关的程序。

这里给出的 \href{https://mirrors.tuna.tsinghua.edu.cn/CTAN/systems/texlive/Images/}{TeXLive}下载路径是清华镜像
\footnote{即摘要中网址\href{https://www.cnblogs.com/fantacity/p/5100434.html}{(a)}给出的TeXLive第一种下载方式}
选择 \texttt{texlive2019.iso}镜像文件
\footnote{该网站会随着TeXLive版本的更新而放出对应的新版本镜像文件,如今是TeXLive2019版本}
% 
\item[(b)]
\textbf{下载 Sublime Text 3}

Sublime Text 2/3是一个跨平台的代码编辑器,同时支持 Windows、 Linux、 Mac OS等操作系统。它具有漂亮的用户界面和强大的功能,例如代码缩略图,Python的插件,代码段等,是我们选用编辑{\LaTeX}文本的编辑器。

百度 Sublime Text进入官网即可下载 \href{https://www.sublimetext.com/3}{Sublime Text 3}。
% 
\item[(c)]
\textbf{下载 Sumatra PDF}

\href{https://www.sumatrapdfreader.org/download-free-pdf-viewer.html}{Sumatra PDF}是一种轻量级PDF阅读器,类型为开放软件,可以在官网上直接找到其下载方式,同时也提供源代码的下载,然而只能在Windows系统下使用。

这里选用 Sumatra PDF作为PDF阅览器的原因是它可以实现{\LaTeX}代码的反向搜索,即双击PDF上的某个位置可以超链接返回{\LaTeX}对应位置,为查错带来便利并且提高了寻找源代码的效率。

\end{enumerate}
% 
\section{安装 TeXLive、 Sublime Text和 Sumatra PDF}
% 
\begin{enumerate}[leftmargin=*]
\item[(a)]
\textbf{安装 TeXLive}

将下载好后的TeXLive镜像文件解压,右键点击 \texttt{install-tl-windows}选择 \texttt{管理员权限安装}。

安装时请自己选择路径
\footnote{尽量选择默认路径,该镜像文件大小约3-3.5GB,若想节省C盘空间则可另寻路径进行安装},
配置选项时小白请全部选择默认选项,尽量保证安装的TeXLive具有全部内容。
% 
\item[(b)]
\textbf{安装 Sublime Text 3和 Sumatra PDF}

双击官网上下载的Sublime Text 3和 Sumatra PDF安装文件即可,这两个软件大小均在10MB左右,可以直接默认安装路径。

安装时请选择 \texttt{Add to explorer context menu},以便打开后续文件。

请注意Sumatra PDF一定要安装在Program Files文件夹而不要安装在“用户”文件夹,否则后面无法进行。
\end{enumerate}

以上软件安装完毕后请记录安装路径以备后续环境配置时使用。
% 
\section{在 Sublime Text 3中安装 Package Control}
% 
由于某些原因,导致 Sublime Text安装插件的 
\href{https://packagecontrol.io/}{官方网站}
经常连接失败从而导致其插件管理 Package Control安装或使用失败
\footnote{其原因可以参考 \url{https://blog.csdn.net/weixin_43948229/article/details/86495993},但该网站提供的解决方法\textbf{无效}},
这也是摘要中给出的网址\textbf{没有更新的部分}。

笔者在经过多次尝试后找到了一个所谓最简单的方法:
\begin{enumerate}[leftmargin=*]
\item[(1)]
在\href{https://github.com/}{GitHub}网站上直接下载 \href{https://github.com/wbond/package_control}{Package Control插件},解压后找到文件 \texttt{Package Control.sublime-settings}。
% 
\item[(2)]
在 Sublime Text 3中打开 \texttt{Preferences->Browse Packages}对应的文件夹,回到上一级路径点开名为 \texttt{Installed Packages}的文件夹,将下载好的 \texttt{Package Control.sublime-settings}文件放入。
% 
\item[(3)]
重启 Sublime Text 3,打开 \texttt{Preferences}可以发现显示 Package Control,说明安装成功。
\end{enumerate}
% 
\section{在 Sublime Text 3中安装 LaTeX Tools等插件并配置}
% 
\begin{enumerate}[leftmargin=*]
\item[(1)]
\textbf{安装 LaTeX Tools插件}

在 Sublime Text 3内按 \texttt{Ctrl+Shift+P},输入 \texttt{install},在窗口内选择 \texttt{Package Control: install package}。

在跳出的窗口中输入 \texttt{latex}查找到 \texttt{LaTeX Tools}点击安装。
% 
\item[(2)]
\textbf{配置 LaTeX Tools}

选择 \texttt{Preferences->Browse Packages}进入文件夹,选择 \texttt{LaTeXTools}文件夹并进入,右键 \texttt{Open with Sublime Text}文件 \texttt{LaTeXTools.sublime-settings}。

利用 \texttt{Ctrl+F}搜寻功能查找 \texttt{windows},选择第二个结果(216行),在220行中将 \texttt{texpath}中的路径改为TeXLive安装路径,并参照219行给出的范例注意路径中的单斜杠变为双斜杠
\footnote{若采取的是默认安装路径,则在220行引号内输入 \texttt{C:$\backslash \backslash$texlive$\backslash \backslash$2019$\backslash \backslash$bin$\backslash \backslash$win32;\$PATH}}。

在222行的 \texttt{distro}后面的引号内改成 \texttt{texlive}。

在224行的 \texttt{sumatra}后面的引号内输入Sumatra PDF的安装路径,书写规则同TeXLive
\footnote{若采取的是默认安装路径,则输入 \texttt{C:$\backslash \backslash$Program Files$\backslash \backslash$SumatraPDF$\backslash \backslash$SumatraPDF.exe}}。

利用 \texttt{Ctrl+F}搜寻功能查找 \texttt{builder},选择位于386行的结果,将引号内内容改成 \texttt{simple}。

按\texttt{Ctrl+S}保存该文件并关闭。
% 
\item[(3)]
\textbf{安装 ConvertToUTF8插件}

ConvertToUTF8插件是一款在 Sublime Text 3中可以识别中文的插件。

类似安装 LaTeX Tools插件,在 Sublime Text 3内按 \texttt{Ctrl+Shift+P},输入 \texttt{install},在窗口内选择 \texttt{Package Control: install package},输入 \texttt{Convert}查找到 \texttt{ConvertToUTF8}点击安装。
\end{enumerate}
% 
\section{配置 Sumatra PDF}
% 
\begin{enumerate}[leftmargin=*]
\item[(1)]
\textbf{添加环境变量}

右键点击 \texttt{我的电脑},选择 \texttt{属性},点击 \texttt{高级系统设置},再点击下方的 \texttt{环境变量}并双击 \texttt{环境变量PATH}。

新建环境变量,将\texttt{SumatraPDF.exe}所在安装目录输入并确定即可。
% 
\item[(2)]
\textbf{配置 Sumatra PDF进行反向搜索}

打开命令行,输入
% 
\begin{lstlisting}[language=tex]
  sumatrapdf.exe -inverse-search "\"C:\Program Files\Sublime Text 3\sublime_text.exe\" \"\%f:%l\
\end{lstlisting}

关闭弹出来的Sumatra PDF窗口即可。
\end{enumerate}
% 
\section{测试}
% 
新建一个{\TeX}文件,用 Sublime Text 3打开,将以下代码输入:
% 
\begin{lstlisting}[language=tex]
  %!TEX program = xelatex
  \documentclass[UTF8]{ctexart}
  \begin{document}
  Hello World!
  \end{document}
\end{lstlisting}

按 \texttt{Ctrl+B}进行编译,编译成功,说明{\LaTeX}环境已搭建成功。
% 
\section{一些说明}
\subsection{关于默认编译方法的选择}
% 
% 
\subsection{关于自动补全}
%
在 Sublime Text 3内按 \texttt{Ctrl+Shift+P},输入 \texttt{install},在窗口内选择 \texttt{Package Control: install package}。

在跳出的窗口中输入 \texttt{latex}查找到 \texttt{LaTeXing}和\texttt{LaTeXYZ}点击安装。
% 
\subsection{关于主题设置}
%  
本人比较喜爱的Sublime Text 3主题为\texttt{Solarized Space Predawn.sublime-text}。

\end{document}